\documentclass[a4paper, 10pt]{article}
    \usepackage[ngerman]{babel}
    \usepackage[utf8]{inputenc}
    \usepackage[T1]{fontenc}
    \usepackage{hyphenat}
    \hyphenation{Mathe-matik wieder-gewinnen}
    \usepackage{amsmath}
    \usepackage{amssymb}
    \usepackage{hyperref}
    \usepackage{svg}
    \usepackage{graphicx}
    \usepackage{rotating}

    \title{Praktikum Grundlagen der Programmierung Hausaufgaben Blatt 10}
    \author{Maximilian Frühauf}
\begin{document}
\maketitle
\begin{itemize}
	\item Welches Statement kompiliert nicht? Wieso kompiliert es nicht?
	      \begin{itemize}
		      \item[] Antwort:
		      \item[] Alle Statements kompilieren.
	      \end{itemize}
	\item Welches Statement kompiliert und wirft eine Exception zur Laufzeit? Wieso wird hier eine Exception geworfen?
	      \begin{itemize}
		      \item[] Antwort:
		      \item[] Statement 7 wirft eine java.lang.ClassCastException zur Laufzeit, da ein Objekt vom
		            Typ \( A \) zu dem spezifischeren Typ \( B \) gecasted wird, dies aber keine Instanz dessen,
		            oder eines Untertypen ist.
	      \end{itemize}
	\item Ausgaben der Statements:
	      \begin{enumerate}
		      \item A.A+
		      \item A.B+
		      \item A.C+
		      \item B.A+
		      \item B.A+
		      \item B.A+
		      \item wirft eine java.lang.ClassCastException
		      \item B.C+
	      \end{enumerate}
\end{itemize}
\end{document}